\begin{frame}{Zaporedja, vrste in limite}{Okolje\texttt{displaystyle} in \texttt{textstyle}}
	\begin{enumerate}
		\item 
		Naj bo  $\sum_{n=1}^\infty{a_n}$ absolutno konvergentna vrsta in $a_n \ne -1$.
		Dokaži, da je tudi vrsta $\sum_{n=1}^\infty \frac{a_n}{1+a_n}$
		absolutno konvergentna.

		\item
		Izračunaj limito
		$$\lim_{x\to\infty} (\sin\sqrt{x+1}-\sin\sqrt{x})$$
		

		\item
		Za dani zaporedji preveri, ali sta konvergentni.
		% Pomagajte si s spodnjima delno pripravljenima matematičnima izrazoma:
	
	\[
	\underbrace{a_n = \sqrt{2+\sqrt{2+\dots+\sqrt{2}}} }_{\text{n korenov}} \qquad
 \underbrace{b_n = \sin(\sin(\dots(\sin 1)\dots))}_{\text{n korenov}}\]
		
	\end{enumerate}
\end{frame}

\begin{frame}{Algebra}{Okolje\texttt{displaystyle} in \texttt{textstyle}}
	\begin{enumerate}
		\item
		Vektorja $\vec{c}=\vec{a}+2\vec{b}$ in $\vec{d}=\vec{a}-\vec{b}$
		sta pravokotna in imata dolžino 1. Določi kot med vektorjema $\vec{a}$ in $\vec{b}$.
		\item 
		Izračunaj

		$\begin{pmatrix}
			1 & 2 & 3 & 4 & 5 & 6\\
			4 & 5 & 2 & 6 & 3 & 1
		\end{pmatrix}^{-2000}$

		
	\end{enumerate}
\end{frame}

\begin{frame}{Velika determinanta}{Okolje\texttt{displaystyle} in \texttt{textstyle}}
	Izračunaj naslednjo determinanto $2n \times 2n$, ki ima na neoznačenih mestih ničle.
	
	$\begin{vmatrix}
		1 &   & 3 & & 1 & & & &\\
		  & 2 & 3 & & 1 & & & &\\
		  &   & 3 & & 1 & & & &\\
		  &   & 3 & n-1 & 1 & & & &\\
		1 & 2 & \dots & n-1 & n & n+1 & n+2 & \dots & 2n\\
		  &   & 3 & & 1 & & & &\\
		  &   & 3 & & 1 & & & &\\
		  &   & 3 & & 1 & & & &
		\end{vmatrix}$

\end{frame}

\begin{frame}{Grupe}{Okolje\texttt{displaystyle} in \texttt{textstyle}}
	Naj bo
	??
	\begin{enumerate}
		\item
			Pokaži, da je $G$ podgrupa v grupi ??
			neničelnih kompleksnih števil za običajno množenje.
		\item
			Pokaži, da je $H$ podgrupa v aditivni grupi ??
			ravninskih vektorjev za običajno seštevanje po komponentah.
		\item
			Pokaži, da je preslikava $f:H\to G$, podana s pravilom
			??
			izomorfizem grup $G$ in $H$.
	\end{enumerate}
\end{frame}


